\chapter{Conclusion}
\label{chap:conclusion}

We have seen how engineers such as Woodland and Silver have built the basis for a barcode system that optimises many processes, and how these barcodes provide an easy, fast and reliable way of encoding data. These values, ease of use, speed and reliability, are most certainly what the development of new technologies is all about.
The sheer number of barcodes currently in use around the world is proof of their ingenuity.
Following the same success, QR-Codes conquered the world and became part of our daily lives as we encounter them everywhere.

Because they have become standard elements in our society, they are often disregarded and wrapped in mystery, like magical tags instantly recognizable by our devices. This work is obviously not an exhaustive list of all codes that exist. Many other types can be commonly found, such as PDF-417 or Aztec codes, both of which also use the Reed-Solomon algorithm for error-correction, or Postnet, a specialized type of barcodes used by the United States Postal Service.

The Lycacode described in \autoref{chap:custom_code} was primarily designed for education purposes, as a way to put in practice the formerly explained principles, but could well have a real application in the Collège.
