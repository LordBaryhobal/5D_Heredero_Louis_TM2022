\chapter{Barcodes}
\label{chap:barcode_origin}

\section{Origin}
\label{sec:barcode_origin}

The patent for the first barcode\cite{barcode_patent} was filed slightly more than 70 years ago, in 1949, by then Drexel University students Norman J. Woodland and Bernard Silver. Officially established in 1952, this patent described the first barcode, a reading device and a circular design. However, success did not strike immediately, mainly because of the impractical and limited resources of the time.
Its first use was on trains, as an identification system (KarTrak\cite{kartrak}). It was soon abandonned however because of readability and maintenance problems.

A few years later, in the 1970s, IBM, which now counted Mr. Woodland as an associate, had developed the linear UPC barcode. The Universal Product Code is still in use today, with some modifications and standardizations. It was and is used in stores to identify groceries. This kind of tagging allowed for shorter waits at registers when checking out and greatly influenced the capitalist society we live in nowadays. As well as improving time efficiency, it also reduced the number of human errors which could happen when manually entering the prices of items bought by customers.

Since then, many other types of barcodes have been created for more specialized purposes, like postal mail, warehouse inventories, libraries or medicines. The main advantage of barcodes is that they can be read very quickly with a single laser scan. The first barcodes didn't even need a computer and could be decoded with only relatively simple electronic circuits \cite[p.3]{barcode_patent}.

In 1974, the GS1 was founded. This international group is responsible for managing encoding standards in the field of logistics and sale of goods.
