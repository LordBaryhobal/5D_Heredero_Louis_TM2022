\begin{titlepage}
  \begin{center}
    \Huge \textbf{Barcodes and QR-Codes}

    \large \textbf{Examples of error detection and correction}

    \vspace{1cm}

    \LARGE Louis Heredero

    \large 4D-5D

    \includegraphics[width=0.7\linewidth]{images/lycacode_ex_final}

    \Large Accompanying teacher: Daniel Erspamer

    \vspace{2cm}

    \LARGE Maturity work 2022-2023
    \vspace{1cm}

    \Large Lycée-Collège de l'Abbaye \\ 1890 St-Maurice

  \end{center}
\end{titlepage}

\clearpage
~
\thispagestyle{empty}
\clearpage

\begin{abstract}
  This work focuses on the creation of barcodes and QR-Codes. It describes and explains the different data encodings and algorithms which make such technologies possible. Following the introduction, the second part is about Code-39 and EAN barcodes, and the third about QR-Codes. Then, the fourth chapter presents in more details some methods for error detection and correction. The final section introduces a new custom type of code named Lycacode which relies upon some aspects seen in the previous three chapters. Additionally, many parts are implemented in Python like the generation of QR-Codes and barcodes for example. These can either be found in the appendices or in the associated \hreffn{https://github.com/LordBaryhobal/5D\_Heredero\_Louis\_TM2022/tree/main/python}{GitHub repository}.
\end{abstract}

\tableofcontents
\listoffigures
\listoftables
